\documentclass[UTF8]{ctexart}
\usepackage[a4paper, margin=0.5in]{geometry}
\usepackage{anyfontsize, tabularx, multirow, hyperref, titlesec, comment, enumitem}
\titleformat{\section}{\Large\scshape\raggedright}{}{0em}{}[\titlerule]
\titlespacing{\section}{0pt}{3pt}{3pt}
\thispagestyle{empty}
\setlist[itemize]{leftmargin=*, before=\vspace{-0.5\baselineskip}, after=\vspace{-\baselineskip}, nosep}
\newcommand{\itemcols}[1]{
\multicolumn{2}{p{\dimexpr \linewidth-2\tabcolsep}}{
\begin{itemize}
#1
\end{itemize}
}
}
\begin{document}
\noindent
\begin{tabularx}{\linewidth}{X r}
\multirow{4}{*}{{\fontsize{50}{60}\selectfont 吴越}}
& \href{mailto:yuw264@ucsd.edu}{yuw264@ucsd.edu} \\
& (+1) (858) 666-5847 \\
& \href{https://github.com/ALMSIVI}{https://github.com/ALMSIVI} \\
& \href{https://www.linkedin.com/in/yue-wu-aaab2213b/}{https://www.linkedin.com/in/yue-wu-aaab2213b/}
\end{tabularx}
\section{教育经历}
\smallskip
\noindent
\begin{tabularx}{\linewidth}{X r}
\textbf{\large 加州大学圣地亚哥分校} & \texttt{2016/09-2021/12} \\
\itemcols{
\item 专业:计算机科学(B.S.与 M.S.) \textbf{(GPA: 3.92/4)}
}
\end{tabularx}
\section{技能}
\smallskip
\noindent
\begin{tabularx}{\linewidth}{l X}
\texttt{语言} & C\#, Java, Javascript/Typescript, HTML, CSS, C/C++, Python, \LaTeX, Shader (GLSL/HLSL) \\
\texttt{框架} & Unity, SteamVR, WPF, Android, jQuery, Node.js, SQL, OpenGL/WebGL, React.js, THREE.js \\
\texttt{软件} & vim, git, Linux
\end{tabularx}
\section{工作经验}
\smallskip
\noindent
\begin{tabularx}{\textwidth}{X r}
\textbf{\large Unity开发者}, UCSD Qualcomm Institute, 加州拉霍亚 & \texttt{2018/10至今} \\
\itemcols{
\item 为美国普华永道的Bodylogical系统开发iOS AR应用。
\item 在Unity中设计并实现了3套不同的可视化模组,数据以时间排列,分布在3维空间中,用户可以方便地理解Body logical的核心功能。
\item 从头编写了一套基于XML的本地化系统与教程系统,使得非美国客户及对AR不熟悉的客户可以轻松上手。
} \\
\textbf{\large 实习前端工程师}, 阿里巴巴集团, 中国杭州 & \texttt{2019/06-2019/08} \\
\itemcols{
\item 基于力导向图原理和Sugiyama算法,用Typescript搭建对图类结构进行3D可视化的库,以替换turf.js。
\item 基于Marching Square算法开发了一套地理模型生成服务,比turf.js自带算法精确100倍,并用于阿里9号馆及双11等数据大屏展示中。
\item 与达摩院的裸眼3D团队合作,研究3维空间下基于Unity/WebGL的人机交互和数据可视化。探索了Entitas框架并编写Shader实现多样的视觉效果。
} \\
\textbf{\large 实习全栈工程师}, 莱曼特信息科技, 中国上海 & \texttt{2018/06-2018/08} \\
\itemcols{
\item 参加软件的改版,从Flash迁移至HTML5。减少代码量,加快运行速度。
\item 完善产品的本地化与响应式UI界面,使其对外国用户更加友好,提升用户体验。
\item 引入JSDoc,对Javascript进行规范,在公司内建立统一的代码风格。
} \\
\end{tabularx}
\section{项目经历}
\smallskip
\noindent
\begin{tabularx}{\linewidth}{X r}
\textbf{\large Exteractive}, 全栈工程师 & \texttt{2020/07至今} \\
\itemcols{
\item 交互小说类型的网页应用,以React作为前端,Express作为后端。
\item 使用MongoDB作为数据库开发了用户系统,用户可以编写故事,续写其他人的故事并对其做出评价。
\item 利用React和Emotion等库实现了响应式UI及本地化。
} \\

\textbf{\large MechSuit VR}, Unity开发者 & \texttt{2017/04-2019/06} \\
\itemcols{
\item 学校虚拟现实俱乐部的小组项目,使用Steam VR和Unity开发基于HTC Vive设备的VR游戏。
\item 实现逆运动学系统,通过VR眼镜和手柄的位置来控制机甲手臂的位置,增加代入感和玩家控制机甲的效率。
\item 编写动力系统,以玩家在物理空间中的移动作为输入,对虚拟空间中的机甲带来推力,玩家无需使用手柄来移动机甲,这种操纵方式给予玩家身临其境之感。
\item 设计了生命值系统与武器系统,并使用MVC模型设计了生命值与弹药显示的UI。利用策略模式等设计模式,编写脚本使得开发人员能迅速增加其他伤害类型与机甲抗性。
} \\

\textbf{\large Transracer}, 全栈工程师 & \texttt{2018/04-2018/06} \\
\itemcols{
\item 4人小组项目,共同开发网页app,前端使用Bootstrap,后端使用Node.js。用户通过翻译歌词来学习外语。
\item 使用sqlite编写所有数据库逻辑,数据库用于存储歌曲与用户得分,用户可将喜欢的歌曲上传到网站以练习。
\item 改善了计算得分逻辑:app会基于用户使用提示的数量及回答的正确率来计算用户得分,使得分更为准确。
} \\

\end{tabularx}
\end{document}