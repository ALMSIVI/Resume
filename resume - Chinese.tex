\documentclass[UTF8]{ctexart}
\usepackage[a4paper, margin=0.5in]{geometry}
\usepackage{anyfontsize, tabularx, multirow, hyperref, titlesec, comment, enumitem}

\titleformat{\section}{\Large\scshape\raggedright}{}{0em}{}[\titlerule]
\titlespacing{\section}{0pt}{3pt}{3pt}

\thispagestyle{empty}

% List style
\setlist[itemize]{leftmargin=*, before=\vspace{-0.5\baselineskip}, after=\vspace{-\baselineskip}, nosep}

% Customized multicolumn
\newcommand{\itemcols}[1]{
	\multicolumn{2}{p{\dimexpr \linewidth-2\tabcolsep}}{
		\begin{itemize}
			#1
		\end{itemize}
	}
}

\begin{document}
	\noindent
	\begin{tabularx}{\linewidth}{X r}
		\multirow{3}{*}{{\fontsize{40}{50}\selectfont 吴越}}
		& \href{mailto:yuw264@ucsd.edu}{yuw264@ucsd.edu} \\
		& +1-(858) 666-5847 \\
		& \href{https://github.com/ALMSIVI}{https://github.com/ALMSIVI}
	\end{tabularx}
	
	
	% If explicitly asked to, uncomment.
	\begin{comment}
	\noindent
	%\textbf{寻求2019年6-9月软件工程方面的实习。}
	\end{comment}
	
	% Header including mailing address
	\begin{comment}
	\begin{tabularx}{\linewidth}{X r}
	\multirow{5}{*}{{\fontsize{45}{60}\selectfont 吴越}} 
	& \href{mailto:yuw264@ucsd.edu}{yuw264@ucsd.edu} \\
	& +1-(858) 666-5847 \\
	& 8840 Costa Verde Blvd. Apt. 3322 \\
	& San Diego, CA 92122
	& United States
	\end{tabularx}
	\end{comment}
	
	
	\section{教育经历}
	\smallskip
	\noindent
	\begin{tabularx}{\linewidth}{X r}
		\textbf{\large 美国加州大学圣地亚哥分校}(University of California, San Diego) & \texttt{2016/09-2020/06} \\
		\itemcols {
			\item 专业:计算机科学(B.S. Computer Science),辅修:视觉艺术(计算与艺术) \textbf{(GPA: 3.90/4)
			%\item 已修课程:数据结构、算法、软件工程、交互设计、图形学、离散数学、计算理论,操作系统
		}
	}
	\end{tabularx}
	
	
	\section{技能}
	\smallskip
	\noindent
	\begin{tabularx}{\linewidth}{l X}
		\texttt{语言} & C\#, Java, HTML, CSS, Javascript, C, C++, Python, \LaTeX \\
		\texttt{技术} & Unity, SteamVR, Windows Desktop, Android, JavaFX, jQuery, Node.js, SQL, OpenGL, React.js \\
		\texttt{软件} & Visual Studio, Android Studio, vim, Blender, git, Linux
	\end{tabularx}


	\section{工作经验}
	\smallskip
	\noindent
	\begin{tabularx}{\textwidth}{X r}
		\textbf{\large 实习前端工程师},莱曼特信息科技,中国上海 & \texttt{2018/06-2018/08} \\
		\itemcols{
			\item 参加软件的改版,从Flash迁移至HTML5。减少代码量,加快运行速度。
			\item 改善产品的英文版本,使其更容易被国外客户所理解。与国外客户会面,向其宣传产品。
			\item 引入JSDoc,对Javascript进行规范,在公司内建立统一的代码风格。
		} 
	\end{tabularx}
	
	
	\section{其他项目}
	\smallskip
	\noindent
	\begin{tabularx}{\linewidth}{X r}
		\textbf{\large Bodylogical AR},Unity开发者 & \texttt{2018/10至今} \\
		\itemcols{
			\item 与UCSD计算机部门的Jurgen Schulze教授一同开发iOS AR应用以展示Bodylogical(一个人体健康数据模拟系统)的功能和应用前景。
			\item 使用C\#设计面板,通过3D带状图(Ribbon chart)显示健康数据,利用AR同时显示多组数据进行对比。
			\item 使用Unity动画模组,开发人物动画,以直观地显示人物的健康状况。
			\item 目前正在开发“内部原理”可视化模组,通过人物内部(器官、循环等)的变化反映健康状况的变化。
		} \\
		
		\textbf{\large MechSuit VR},Unity开发者 & \texttt{2017/04至今} \\
		\itemcols{
			\item 5人小组项目,使用Steam VR和Unity编写VR游戏。玩家可以穿着格斗机甲,在竞技场中战斗。
			\item 开发了适用于机甲的逆运动学系统(inverse kinematic system)。通过追踪玩家头部(VR眼镜)和手部(VR控制器)的位置,机甲的手臂会根据玩家手臂而移动,使得玩家可以高效控制机甲。
			\item 编写了动力系统,玩家可以在现实空间中摆动身体以实现在虚拟空间中的位移。相比控制摇杆,这种操纵方式给予玩家身临其境之感。
			\item 设计了生命与武器系统,以及配套UI来显示玩家生命与当前武器。使用策略(Strategy)模式编写脚本,使其能够被轻易扩展,可支持各种伤害类型与伤害抗性。
			\item 目前正在利用Tensorflow的深度学习功能开发玩家动作识别系统,以实现武器切换等多种功能。
		} \\
		
		\textbf{\large Transracer},全栈开发者 & \texttt{2018/04-2018/06} \\
		\itemcols{
			\item 4人小组项目,共同开发网页app,前端使用Bootstrap,后端使用Node.js。用户通过翻译歌词来学习外语。
			\item 使用sqlite编写所有数据库逻辑,数据库用于存储歌曲与用户得分,用户可将喜欢的歌曲上传到网站以练习。
			\item 改善了计算得分逻辑:app会基于用户使用提示的数量及回答的正确率来计算用户得分,使得分更为准确。
		} \\
		
		\textbf{\large Flashback Music},安卓开发者 & \texttt{2018/01-2018/03} \\
		\itemcols{
			\item 5人小组项目,遵循敏捷(Agile)开发原则,开发安卓程序。这款音乐app会记录用户听歌的时间和地点。
			\item 使用Model-View-Controller模式,分离UI交互与后台逻辑,使得代码更加清晰。
			\item 使用策略(Strategy)模式和工厂(Factory)模式,为本地音乐和云端音乐设计了不同的逻辑,并将其统一在同一套接口(interface)之下。
		}
	\end{tabularx}

\begin{comment}
		\textbf{\large WayAround},前端开发者 & \texttt{2018/01-2018/03} \\
\itemcols{
\item 4人小组项目,共同开发网页程序的视觉原型(mockup)。该导航软件允许用户根据喜好定义线路。
\item 在完成纸质原型后,使用Bootstrap和jQuery开发了首页。编写了两版首页以供A/B测试,分析用户选择喜好所花费的时间,以了解用户如何接触一款新产品,并根据结果修改首页使其对用户更加友好。
\item 完善评价页,根据用户之前选择的喜好动态生成表单,使页面更为精简。	
} \\
\end{comment}
\end{document}