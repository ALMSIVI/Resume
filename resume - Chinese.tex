\documentclass[10pt, letterpaper]{ctexart}
\usepackage[T1]{fontenc}
\usepackage[letterpaper, margin=0.5in]{geometry}
\usepackage{anyfontsize, tabularx, multirow, hyperref, titlesec, comment, enumitem}
\titleformat{\section}{\Large\scshape\raggedright}{}{0em}{}[\titlerule]
\titlespacing{\section}{0pt}{3pt}{3pt}
\thispagestyle{empty}
\setlist[itemize]{leftmargin=*, noitemsep, topsep=2pt, parsep=1pt}
\setlength \parindent {0pt}
\begin{document}
	\begin{tabularx}{\linewidth}{X r}
		\multirow{5}{*}{{\fontsize{50}{60}\selectfont 吴越}}
		& \href{mailto:yuw264@ucsd.edu}{yuw264@ucsd.edu} \\
		& (+1) (858) 666-5847 \\
		& \href{https://github.com/ALMSIVI}{https://github.com/ALMSIVI} \\
		& \href{https://www.linkedin.com/in/yuewu-almsivi}{https://www.linkedin.com/in/yuewu-almsivi}
	\end{tabularx}
	\section{教育经历}
	\smallskip
	\textbf{\large 加州大学圣地亚哥分校}
	\begin{itemize}
		\item 专业:计算机科学(B.S) \textbf{(GPA: 3.92/4)} \hfill \texttt{2016/09-2020/06}
		\item 专业:计算机科学(M.S.) \textbf{(GPA: 4/4)} \hfill \texttt{2020/09-2021/12}
	\end{itemize}
	\section{技能}
	\smallskip
	\begin{tabularx}{\linewidth}{l X}
		\texttt{语言} & C\#, Java, Javascript/Typescript, HTML, CSS, C/C++, Python, \LaTeX, Shader (GLSL/HLSL) \\
		\texttt{框架} & Unity, SteamVR, WPF, Android, jQuery, Node.js, SQL, OpenGL/WebGL, React.js, THREE.js \\
		\texttt{软件} & vim, git, Linux
	\end{tabularx}
	\section{工作经验}
	\smallskip
	\textbf{\large 研究助理}, UCSD校医院, \textit{加州圣地亚哥} \hfill \texttt{2020/10至今}
	\begin{itemize}
		\item 进行关于精神分裂症的研究项目,将患者置于VR环境之下,研究其大脑活动。

		\item 优化已有的VR应用以满足时间和测量的要求,将音频延迟从高于100ms降低至稳定的35ms。

		\item 创建新的VR场景,增加音频样式,使得研究员可以进行更加详细的测试,获取更好的数据。

	\end{itemize}
	\textbf{\large Unity开发者}, UCSD Qualcomm Institute, \textit{加州拉霍亚} \hfill \texttt{2018/10至今}
	\begin{itemize}
		\item 为美国普华永道的Bodylogical系统开发iOS AR应用。

		\item 在Unity中设计并实现了3套不同的可视化模组,数据以时间排列,分布在3维空间中,用户可以方便地理解Body logical的核心功能。

		\item 从头编写了一套基于XML的本地化系统与教程系统,使得非美国客户及对AR不熟悉的客户可以轻松上手。

	\end{itemize}
	\textbf{\large 实习前端工程师}, 阿里巴巴集团, \textit{中国杭州} \hfill \texttt{2019/06-2019/08}
	\begin{itemize}
		\item 基于力导向图原理和Sugiyama算法,用Typescript搭建对图类结构进行3D可视化的库,以替换turf.js。

		\item 基于Marching Square算法开发了一套地理模型生成服务,比turf.js自带算法精确100倍,并用于阿里9号馆及双11等数据大屏展示中。

		\item 与达摩院的裸眼3D团队合作,研究3维空间下基于Unity/WebGL的人机交互和数据可视化。探索了Entitas框架并编写Shader实现多样的视觉效果。

	\end{itemize}
	\textbf{\large 实习全栈工程师}, 莱曼特信息科技, \textit{中国上海} \hfill \texttt{2018/06-2018/08}
	\begin{itemize}
		\item 参加软件的改版,从Flash迁移至HTML5。减少代码量,加快运行速度。

		\item 完善产品的本地化与响应式UI界面,使其对外国用户更加友好,提升用户体验。

		\item 引入JSDoc,对Javascript进行规范,在公司内建立统一的代码风格。

	\end{itemize}
	\section{项目经历}
	\smallskip
	\textbf{\large Exteractive}, 全栈工程师 \hfill \texttt{2020/07-2020/10}
	\begin{itemize}
		\item 交互小说类型的网页应用,以React作为前端,Express作为后端。

		\item 使用MongoDB作为数据库开发了用户系统,用户可以编写故事,续写其他人的故事并对其做出评价。

		\item 利用React和Emotion等库实现了响应式UI及本地化。

	\end{itemize}
	\textbf{\large MechSuit VR}, Unity开发者 \hfill \texttt{2017/04-2019/06}
	\begin{itemize}
		\item 学校虚拟现实俱乐部的小组项目,使用Steam VR和Unity开发基于HTC Vive设备的VR游戏。

		\item 实现逆运动学系统,通过VR眼镜和手柄的位置来控制机甲手臂的位置,增加代入感和玩家控制机甲的效率。

		\item 编写动力系统,以玩家在物理空间中的移动作为输入,对虚拟空间中的机甲带来推力,玩家无需使用手柄来移动机甲,这种操纵方式给予玩家身临其境之感。

		\item 设计了生命值系统与武器系统,并使用MVC模型设计了生命值与弹药显示的UI。利用策略模式等设计模式,编写脚本使得开发人员能迅速增加其他伤害类型与机甲抗性。

	\end{itemize}
\end{document}
